\documentclass[11pt]{article}
\title{Curriculum Vitae}
\author{Julia Wolf modified}
\date{}
\usepackage{cmbright}

\usepackage[french, english]{babel}
%\usepackage[utf8]{inputenc}

\usepackage{amsmath, wrapfig}
\usepackage{dsfont, a4wide, amsthm, amssymb, amsfonts, graphicx}
\usepackage{fancyhdr, xspace, psfrag, setspace, supertabular, color}
%\usepackage[english,french]{babel}
%\usepackage{showkeys}

\let\oldthebibliography=\thebibliography
  \let\endoldthebibliography=\endthebibliography
  \renewenvironment{thebibliography}[1]{%
    \begin{oldthebibliography}{#1}%
      \setlength{\parskip}{0ex}%
      \setlength{\itemsep}{0ex}%
  }%
  {%
    \end{oldthebibliography}%
  }

\newenvironment{mitemize}{
\begin{itemize}
  \setlength{\itemsep}{1pt}
  \setlength{\parskip}{0pt}
  \setlength{\parsep}{0pt}
}{\end{itemize}}


\usepackage[left=2.2cm, right=2.2cm, top=2.2cm, bottom=2.2cm]{geometry}
\usepackage{sectsty}
\sectionfont{\normalsize}
\pagestyle{plain}

\newtheorem{theorem}{Theorem}[section]
\newtheorem{proposition}[theorem]{Proposition}
\newtheorem{lemma}[theorem]{Lemma}
\newtheorem{claim}[theorem]{Claim}
\newtheorem{corollary}[theorem]{Corollary}
\newtheorem{conjecture}[theorem]{Conjecture}
\newtheorem{definition}[theorem]{Definition}
\newtheorem{problem}[theorem]{Problem}
\newtheorem{example}[theorem]{Example}
\newtheorem{question}[theorem]{Question}
\newtheorem{remark}[theorem]{Remark}

\setlength{\parindent}{0cm}
%\onehalfspacing

\def\eps{\epsilon}
\def\E{\mathbb{E}}
\def\Z{\mathbb{Z}}
\def\R{\mathbb{R}}
\def\T{\mathbb{T}}
\def\C{\mathbb{C}}
\def\N{\mathbb{N}}
\def\P{\mathbb{P}}
\def\F{\mathbb{F}}
\def\Q{\mathbb{Q}}
\def\f{\mathbf{f}}
\def\h{\mathbf{h}}
\def\m{\mathbf{m}}
\def\n{\mathbf{n}}
\def\r{\mathbf{r}}
\def\w{\mathbf{w}}
\def\x{x}
\def\y{\mathbf{y}}
\def\z{\mathbf{z}}
\def\a{\alpha}
\def\b{\beta}
\def\g{\gamma}
\def\d{\delta}
\def\e{\epsilon}

\def\mmu{\boldsymbol{\mu}}
\def\llambda{\boldsymbol{\lambda}}
\def\oomega{\boldsymbol{\omega}}
\def\absmu{\tilde{\mu}}

\def\Lsys{\mathcal{L}}
\def\bone{\mathcal{B}_1}
\def\btwo{\mathcal{B}_2}
\def\B{\mathcal{B}}
\def\I{\mathcal{I}}
\def\X{\mathcal{X}}
\def\Y{\mathcal{Y}}
\def\CF{\mathcal{F}}
\def\K{\mathcal{K}}
\def\Zone{\mathcal{Z}_1}
\def\Ztwo{\mathcal{Z}_2}
\def\Zk{\mathcal{Z}_k}
\def\Zkmo{\mathcal{Z}_{k-1}}
\def\CL{\mathcal{CL}}

\def\Linfmu{L^{\infty}(\mu)}
\newcommand{\snorm}[1]{\lvert\!|\!| #1|\!|\!\rvert}
\newcommand{\type}[1]{^{[#1]}}

\def\ni{\noindent}
\def\iff{\;\;\Leftrightarrow\;\;}
\def\implies{\;\;\Rightarrow\;\;}
\def\mymod{\mbox{ mod }}
\def\tends{\rightarrow}
\def\maps{\rightarrow}
\def\ra{\rightarrow}
\def\seq#1#2{#1_1,\dots,#1_#2}
\def\sm#1#2{\sum_{#1=1}^#2}
\def\sp#1{\langle #1\rangle}
\def\ol{\overline}
\def\hf{\hat{f}}
\def\hQ{\hat{Q}}

\def \codim{{\rm codim}}


\begin{document}

\hline

\begin{center}\large{Curriculum Vitae -- Julia Wolf} \end{center}

\hline

%\maketitle
%\tableofcontents
%\setcounter{section}{-1}


\begin{table}[ht]
%\hspace{0.5cm}
\begin{flushleft}\hspace{-9pt}
\begin{tabular}[t]{ll}
\vspace{3pt}\textbf{Full name} & \textbf{Current position}\\
\vspace{7pt}Julia Wolf & Heilbronn Reader in Combinatorics and Number Theory\vspace{-8pt}\\ & Associate Chair, Heilbronn Institute for Mathematical Research\vspace{5pt}\\
\vspace{3pt}\textbf{Postal address} & \textbf{Contact details}\\
School of Mathematics \hspace{20pt}\;& Office: +44 (0)117 331 1663\\
University of Bristol & Mobile: +44 (0)7455 289 519\\
Bristol, BS8 1TW & Email: julia.wolf@bristol.ac.uk\\
United Kingdom & Web: http://www.juliawolf.org \\
\end{tabular}
\end{flushleft}
\end{table}
\vspace{-13pt}

\textbf{Employment}
\begin{mitemize}

\item 2013--present: \textit{Heilbronn reader in combinatorics and number theory}, University of Bristol.
\item 2013--2015: \textit{Associate professor} (with tenure), \'Ecole polytechnique, Paris. [on leave]
\item 2010--2013: \textit{Hadamard associate professor}, \'Ecole polytechnique, Paris.
\item 2008--2010: \textit{Triennial assistant professor}, Rutgers University, New Brunswick.
\item 2008 (autumn): \textit{Postdoctoral fellow}, Mathematical Sciences Research Institute, Berkeley.
\item 2008 (summer): \textit{SPUR mentor} and \textit{MITES instructor}, MIT, Cambridge, MA.
\item 2007--2008: \textit{Member}, Institute for Advanced Study, Princeton.
\end{mitemize}

\vspace{2pt}
\textbf{Education}
\begin{mitemize}
\item 2012: \textit{Habilitation \`a Diriger des Recherches (HDR)}\\
Universit\'e Paris-Sud (Orsay), examining committee: E. Breuillard, B. Green, E. Fouvry, H. Helfgott, B. Host, A. Plagne.
\item 2003--2008: \textit{Doctor of Philosophy (PhD)}\\
Clare College, University of Cambridge, under the supervision of W.T. Gowers, FRS.
\item 2002--2003: \textit{C.A.S.M. (Part III of the Mathematical Tripos)}\\
Clare College, University of Cambridge, with Distinction.
\item 1999--2002: \textit{Bachelor of Arts (BA Hons)}\\
Clare College, University of Cambridge, in Mathematics (First Class).
\end{mitemize}

%\vspace{3pt}
%\textbf{Languages}
%\vspace{5pt}\\
%English, French, German and Spanish fluent, basic Arabic, Latin diploma.

\vspace{2pt}
\textbf{Visiting positions}
\begin{mitemize}
\item 2017 (spring): Simons Institute for the Theory of Computing, Berkeley.
\item 2013 (autumn): Simons Institute for the Theory of Computing, Berkeley.
\item 2011 (autumn): Erwin Schr\"odinger Institute, Vienna.
\item 2011 (spring): Isaac Newton Institute, Cambridge, UK.
\item 2006 (spring): MIT, Cambridge, MA, visiting graduate student with B. Green.
\item 2005 (autumn): Universitat Polit\`ecnica de Catalunya, Barcelona, Marie Curie pre-doc with O. Serra.
\end{mitemize}

\vspace{2pt}
\textbf{Academic awards}
\begin{mitemize}
\item 2016: \textit{Anne Bennett Prize}, London Mathematical Society.
\item 2015: shortlisted as 1 in 3 for \textit{Outstanding Teaching Award}, Faculty of Science, University of Bristol.
\item 2008: \textit{Hartley Rogers Jr. Prize}, Massachusetts Institute of Technology.
\item 2003: \textit{Robins Prize}, Clare College.
\item 2002: \textit{Harry Paten Scholarship} and \textit{Owst Prize for Mathematics}, Clare College.
\end{mitemize}

\vspace{2pt}
\textbf{Grants}
\begin{mitemize}
\item 2016: Heilbronn Institute for Mathematical Research \textit{Focused Research Grant}, with T. Bloom (GBP 7.5K).
\item 2015: semester-long research programme grant on \emph{Pseudorandomness} at the Simons Institute for the Theory of Computing, Berkeley, to be held Spring 2017, with J. Fox, B. Green, R. Impagliazzo, L. Trevisan and D. Zuckerman (USD 430K).
\item 2014: LMS conference grant \textit{Celebrating new appointments} (GBP 600).
\item 2012--2015: grant from the \textit{Agence Nationale de la Recherche, programme Blanc} CAESAR, with E. Breuillard, A. Granville, H. Helfgott, A. Plagne et al. (EUR 203K)
\item 2003--2007: \textit{Gates Cambridge Scholarship}, Bill and Melinda Gates Foundation.
\end{mitemize}

\vspace{3pt}
\textbf{Research interests}
\vspace{7pt}

My research focuses on identifying and quantifying arithmetic structure in dense sets of integers and combines Fourier-analytic, combinatorial and probabilistic methods. It forms part of a thriving area called arithmetic combinatorics, a subject that includes many beautiful results such as the Green-Tao Theorem on long arithmetic progressions in the primes. Some of my work has close connections with ergodic theory and applications to theoretical computer science.


%\vspace{7pt}
%\textbf{In preparation}
%\begin{mitemize}
%
%\end{mitemize}

\vspace{3pt}
\textbf{Journal articles and conference proceedings}
\begin{mitemize}
\item \textit{Some applications of relative entropy in additive combinatorics.}
\\To appear, 2017.
\item \textit{Ramsey multiplicity of linear patterns in certain finite abelian groups}, with A. Saad.
\\Q. J. Math. 68 (1): 125--140, 2017. [doi: 10.1093/qmath/haw011]
\item \textit{Finite field models in arithmetic combinatorics--ten years on}. 
\\Finite Fields and Their Applications, Vol. 32, 233--274, 2015. [doi:10.1016/j.ffa.2014.11.003]
\item \textit{Sampling-based proofs of almost-periodicity results and applications}, with E. Ben-Sasson, N. Ron-Zewi and M. Tulsiani. 
\\Proceedings of 41st International Colloquium on Automata, Languages and Programming (ICALP), Part I, 955--966, 2014. $[$doi:10.1007/978-3-662-43948-7]
\item \textit{Polynomial configurations in the primes}, with Th.H. L\^e. 
\\Int. Math. Res. Not., first published online August 2013 [doi:10.1093/imrn/rnt169]
\item \textit{Arithmetic and polynomial progressions in the primes, d'apr\`es Gowers, Green, Tao and Ziegler.} 
\\S\'eminaire Bourbaki, 64\`eme ann\'ee, Expos\'e no.1054, Ast\'erisque No. 352: 389--427, 2013.
\item \textit{Quadratic Goldreich-Levin theorems}, with M. Tulsiani. 
\\Proceedings of the 52nd IEEE Symposium on Foundations of Computer Science (FOCS), 619--628, 2011. $[$10.1109/FOCS.2011.59]
\\By invitation in SIAM J. Comput., Vol. 43, No. 2, 730--766, 2014. [doi:10.1137/12086827X]
\item \textit{Linear forms and quadratic uniformity in $\Z_{N}$}, with W.T. Gowers. 
\\J. Anal. Math. 115: 121--186, 2011. [doi:10.1007/s11854-011-0026-7] 
\item \textit{Linear forms and higher-degree uniformity in $\F_{p}^{n}$}, with W.T. Gowers. 
\\Geom. Funct. Anal. 21 (1): 36--69, 2011. [doi:10.1007/s00039-010-0106-3] 
\item \textit{A note on Elkin's improvement of Behrend's construction}, with B. Green. 
\\Additive Number Theory: Festschrift in Honor of the 60th Birthday of Melvyn B. Nathanson, Springer, 2010. $[$doi:10.1007/978-0-387-68361-4]
\item \textit{Subsets of $\F_{q}^{n}$ not containing $k$-term arithmetic progressions}, with Y. Lin.
\\European J. Combin., 31(5):1398--1403, 2010. [doi:10.1016/j.ejc.2009.12.001]
\item \textit{Linear forms and quadratic uniformity in $\F_{p}^{n}$}, with W.T. Gowers. 
\\Mathematika 57 (2): 215--237, 2012. [doi:10.1112/S0025579311001264]
\item \textit{The minimum number of monochromatic 4-term progressions}. 
\\J. Comb. 1(1):53--68, 2010. $[$doi:10.1016/j.ejc.2009.12.001]
\item \textit{The true complexity of a system of linear equations}, with W.T. Gowers. 
\\Proc. London Math. Soc., 100(1): 155--176, 2010. [doi:10.1112/plms/pdp019] 
\item \textit{The structure of popular difference sets}. 
\\Israel J. Math., 179(1): 253--278, 2010. $[$doi:10.1007/s11856-010-0081-2]
%\item An extension of the fundamental theorem on right-angled triangles, with A. Vella and D. Vella. The Mathematical Gazette, Vol. 89, No. 515, 237--244, 2005. [http://www.jstor.org/stable/3621222]
\end{mitemize}

\vspace{3pt}
\textbf{Dissertations}
\begin{mitemize}
\item \textit{L'analyse harmonique d'ordre sup\'erieur, et applications}. Habilitation thesis, December 2012.
\item \textit{Arithmetic structure in sets of integers}. Ph.D. thesis, December 2007. 
\item \textit{Arithmetic structure in difference sets}. Part III essay, May 2003.
\end{mitemize}

\vspace{3pt}
\textbf{Books}
\begin{mitemize}
\item \textit{An introduction to arithmetic combinatorics}. Under contract by Oxford University Press since April 2015, expected completion by July 2017.
\end{mitemize}

\vspace{3pt}
\textbf{Organisation of seminars and conferences}
\begin{mitemize}
\item Co-organiser (chair) of a semester-long research programme on \emph{Pseudorandomness} at the Simons Institute for the Theory of Computing, Berkeley, to be held Spring 2017, with J. Fox, B. Green, R. Impagliazzo, L. Trevisan and D. Zuckerman.
\item Co-organiser of a Focused Research Group on \emph{Recent breakthroughs using the polynomial method}, 19th--23rd September 2016, with T. Bloom.
\item Co-organiser of the \emph{BMC 2016 Special Session in Combinatorics}, with T. Bloom.
\item Co-organiser \emph{Combinatorics meets ergodic theory}, workshop at Banff International Research Station, 20th--24th July 2015, with B. Kra and N. Frantzikinakis.
\item Co-organiser \emph{Bristol-Oxford 1-day meeting in additive combinatorics}, University of Bristol, biannually since September 2014, initially supported by a Celebrating New Appointments grant of the LMS, now with B. Green and T. Browning.
\item Co-organiser (chair) \emph{Finding algebraic structures in extremal combinatorial configurations}, workshop at IPAM, Los Angeles, 19th--23rd May 2014, with E. Breuillard, B. Green, J. Solymosi and T. Tao.
\item Co-organiser \textit{Neo-classical methods in discrete analysis}, workshop at the Simons Institute for the Theory of Computing, Berkeley, 2nd--6th December 2013, with L. Trevisan.
%\item Co-organiser \textit{Combinatorics seminar}, University of Bristol, 2013--present.
\item Co-organiser \textit{Additive Combinatorics in Paris}, international conference with 130 participants at the Institut Henri Poincar\'e in Paris, 9th--13th July 2012, with A. Plagne, E. Balandraud, B. Girard and W. Schmid. 
\item Co-organiser \textit{Journ\'ee Combinatoire et Informatique}, 1-day colloquium at \'Ecole polytechnique, 4th July 2012, with B. Charron-Bost.
%Responsibilities include scientific programme, all communications (website, poster, announcements, publicity), database management.
\item Organiser \textit{Groupe de travail en combinatoire arithm\'etique}, IHP, Paris, 2011--2013.
\item Co-organiser \textit{Discrete math seminar} at Rutgers University, spring and autumn 2009, with V. Vu.
\item Co-organiser \textit{Analysis seminar} at the Institute for Advanced Study, spring 2008, with T. Sanders.
\end{mitemize}

\newpage 
\textbf{Talks for a general mathematical audience}
\begin{mitemize}
\item June 2018: \textit{British Mathematics Colloquium 2018}, public lecture, University of St Andrews.
\item April 2017: \textit{Colloquium}, Center for Communications Research, La Jolla, CA.
\item April 2017: \textit{Open lecture}, Simons Institute, University of California Berkeley.
\item February 2016: \textit{Mathematics Department Colloquium}, Warwick University.
\item[$\star$] December 2015: \textit{Sofia Kovalevskaya Colloquium}, Berlin Mathematical School.
\item November 2015: \textit{Pure Mathematics Colloquium}, University of Sheffield.
\item February 2014: \textit{Landscapes in Mathematical Sciences}, University of Bath.
\item[$\star$] November 2012: \textit{Forum des jeunes mathŽmaticien-ne-s}, plenary lecture, IHP, Paris. 
\item May 2012: \textit{Colloquium de l'Institut de Math\'ematiques de Jussieu}, Paris.
\item April 2012: \textit{Young Researchers in Mathematics 2012}, plenary lecture, University of Bristol.
\item March 2012: \textit{S\'eminaire Bourbaki}, IHP, Paris.
\item January 2010: \textit{Mathematics Department Colloquium}, UG Athens, GA.
\item[$\star$] October 2009: \textit{Women in Mathematics Lecture Series}, MIT, Cambridge, MA.
\item October 2008: \textit{Mathematics Department Colloquium}, UC Santa Cruz, CA. 
\end{mitemize}

\textbf{Invited lecture series}
\begin{mitemize}
\item August 2015: \emph{AGRA 2015: Summer school on arithmetic, groups and analysis} [in Spanish], CIMPA, Cusco, Peru.
\item October 2014: \emph{Additive Combinatorics Doctoral School} [in English], Freie Universit\"at Berlin.
\item June 2012: \emph{Colloque Jeunes Chercheurs en Th\'eorie des Nombres} [in French], ENS Lyon.
\item January 2012: \emph{First French-Chilean Congress in Dynamics and Combinatorics} [in English], Saint-Valery-sur-Somme, France. 
\end{mitemize}

\vspace{3pt}
\textbf{Invited conference presentations}
\begin{mitemize}
\item June 2018: \textit{Discrete Mathematics Days/JMDA18}, plenary talk, Seville, Spain.
\item May 2018: \textit{Georgia Discrete Analysis}, University of Georgia, Athens, GA.
\item December 2017: \textit{10th Anniversary of the Centre for Discrete Mathematics and its Applications (DIMAP)}, University of Warwick.
\item October 2017: \textit{Additive Combinatorics and Applications}, Harvard University, Cambridge, MA.
\item September 2017: \textit{The music of numbers--in memory of Javier Cilleruelo}, ICMAT, Madrid.
\item July 2017: \textit{British Combinatorial Conference}, plenary talk, University of Strathclyde.
\item June 2017: \textit{Interactions with Combinatorics}, University of Birmingham.
\item June 2017: \textit{Canadam}, plenary talk, Ryerson University, Toronto.
\item June 2017: \textit{13th International Conference on Finite Fields and their Applications}, Gaeta, Italy.
\item September 2015: \textit{Joint LMS/EMS Mathematical Weekend}, University of Birmingham.
\item September 2015: \textit{Non-combinatorial combinatorics}, University of Warwick.
\item July 2015: \textit{Journ\'ees Arithm\'etiques 2015}, plenary talk, Debrecen, Hungary.
\item July 2015: \textit{Sextas Jornadas de Teor\'ia de Numeros}, plenary talk, Valladolid, Spain.
\item April 2015: \textit{Postgraduate Combinatorics Conference}, plenary talk, Queen Mary, London. 
\item March 2015: \textit{British Mathematical Colloquium}, combinatorics session, University of Cambridge.
\item November 2014: \textit{Arithmetic and allied subjects on the banks of the Neva}, Euler International Mathematical Institute, St Petersburg. 
%\item July 2014: \textit{Emerging Leaders and Evolving Frontiers in Analytic Number Theory}, Hausdorff Center for Mathematics, Bonn.
\item June 2014: \textit{Second Joint International Meeting of the IMU and the AMS}, Tel Aviv.
\item March 2014: \textit{Simons Symposium: Beyond the Boolean cube}, Puerto Rico.
\item December 2013: \textit{Emerging applications of finite fields}, Linz.
\item November 2013: \textit{Harmonic analysis/PDEs meeting}, University of Birmingham.
\item May 2013: \textit{London 2-day colloquium in combinatorics}, London School of Economics.
\item March 2013: \textit{Number theory days 2013}, EPFL, Lausanne.
\item April 2012: Mini-workshop on \textit{Hypergraph Turan Problems}, Oberwolfach.
\item October 2011: \textit{Integers 2011}, plenary lecture, West Carollton, GA.
\item June 2011: \textit{Journe\'e de Th\'eorie Analytique des Nombres}, Nancy Universit\'e, France.
%\item February 2011: \textit{Discrete Methods in Ergodic Theory}, workshop on the occasion of T. Tao's Frederic Esser Nemmers Prize, Northwestern University, IL [cancelled due to illness]. 
\item November 2010: \textit{Exponential sums over finite fields and applications}, ETH, Z\"urich. 
\item December 2009: \textit{Analytical Methods in Combinatorics, Additive Number Theory \\and Computer Science}, IPAM, Los Angeles, CA.
\item November 2009: AMS Sectional Meeting, Special Session \textit{Arithmetic Combinatorics}, \\UC Riverside, CA.
\item October 2009: AMS Sectional Meeting, Special Session \textit{Analytic Number Theory}, \\Penn State University, State College, PA.
\item May 2009: \textit{Dynamical Systems and Randomness}, IHP, Paris.
\item November 2008: \textit{Discrete Rigidity Phenomena in Additive Combinatorics}, MSRI, Berkeley. 
\item July 2008: \textit{Canadian Number Theory Association Meeting X}, Waterloo, ON. 
\end{mitemize}

%\textbf{Job Talks}
%\begin{mitemize}
%\item June 2010: ETH, Z\"urich, Switzerland. 
%\item May 2010: \'Ecole polytechnique, Paris, France. 
%\item May 2010: University of Z\"urich, Switzerland.
%\end{mitemize}

\textbf{Selected recent seminar talks}
\begin{mitemize}
\item April 2017: \emph{Combinatorics seminar}, UC Berkeley.
\item September 2016:  \emph{Rencontres de Th\'eorie Analytique et El\'ementaire des Nombres}, IHP, Paris. 
\item December 2015: \emph{Discrete mathematics seminar}, Freie Universit\"at Berlin.
\item November 2014: \emph{DIMAP seminar}, University of Warwick. 
\item January 2014: \emph{S\'eminaire G\'eometrie}, Institut de Math\'ematiques de Bordeaux.
\item November 2013: \emph{Number theory seminar}, University of Oxford.
\item June 2013: \emph{S\'eminaire Codage, Cryptologie, Algorithmes}, INRIA Saclay, Paris.
\item June 2013: \emph{Algorithms and Complexity Seminar}, LIAFA, Universit\'e Paris 7.
%\item March 2013: Linfoot Seminar, University of Bristol.
\item January 2013: \emph{Discrete analysis seminar}, University of Cambridge.
\item January 2012: \emph{S\'eminaire d'analyse harmonique}, Universit\'e Paris-Sud (Orsay).
\item November 2011: \emph{London-Paris Number Theory Seminar}, IHP, Paris.
\item October 2011: \emph{Theory Seminar}, Department of Computer Science, University of Chicago.
%\item October 2011: Combinatorics and Probability Seminar, University of Pennsylvania, Philadelphia, PA.
%\item June 2011: S\'eminaire tournant de Th\'eorie Analytique des Nombres et d'Approximation Diophantienne, Universit\'e Claude Bernard, Lyon, France. 
\item April 2011: \emph{Heilbronn seminar}, University of Bristol. 
%\item March 2011: S\'eminaire de Th\'eorie Ergodique, Jussieu, Paris, France. 
%\item March 2011: Number Theory Student Seminar, CUNY Graduate Center, New York, NY. 
\item November 2010: \emph{Rencontres de Th\'eorie Analytique et El\'ementaire des Nombres}, IHP, Paris. 
%\item October 2010: S\'eminaire d'Analyse Fonctionelle, Jussieu, Paris.
%\item May 2010: S\'eminaire de Combinatoire Additive, Chevaleret, Paris, France. 
\item February 2010: \emph{Algebra, Number Theory and Combinatorics Seminar}, UT Austin.
\item February 2010: \emph{Discrete Math and Computer Science Seminar}, IAS, Princeton.
\item October 2009: \emph{Discrete Math Seminar}, Princeton University, Princeton.
\item October 2009: \emph{Discrete Math Seminar}, Columbia University, New York.
%%\item March 2009: Discrete Math Seminar, University of Delaware, Newark, DE.
%%\item February 2009: Discrete Math Seminar, Rutgers University, New Brunswick, NJ.
%%\item December 2008: Number Theory Seminar, University of Edinburgh, Edinburgh. 
%%\item September 2008: Number Theory Seminar, CUNY, New York, NY. 
%%\item September 2008: Postdoctoral Seminar, MSRI, Berkeley, CA.
%\item March 2008: NYU-Columbia-CUNY Number Theory Seminar, New York, NY.
%\item March 2008: Stanford/AIM Number Theory Seminar, Palo Alto, CA.
%\item January 2008: PDEs/Analysis Seminar, UC Los Angeles, CA.
%%\item November 2007: Arithmetic Combinatorics Seminar, IAS, Princeton, NJ. 
%\item October 2007: Dynamical Systems Seminar, Northwestern University, Evanston, IL. 
%%\item October 2007: Postdoctoral Seminar, Institute for Advanced Study, Princeton, NJ. 
%\item May 2007: Discrete Analysis Seminar, Cambridge, UK. 
%%\item April 2006: Number theory study group, MIT, Cambridge, MA.
\end{mitemize}

\vspace{3pt}
\textbf{Research supervision}
\begin{mitemize}
\item 2014--present: supervising \textit{doctoral students} P.-Y. Bienvenu and L. Rimanic, University of Bristol.
\item 2014--present: mentoring \textit{postdoctoral fellow} T. Bloom, University of Bristol.
\item 2014: supervised \textit{summer undergraduate research project} of A. Saad, University of Bristol.
\item 2013: supervised \textit{masters thesis} of P.-Y. Bienvenu (Master 2), \'Ecole normale sup\'erieure.
\item 2012--2013: mentored \textit{postdoc} P. Candela, joint with H. Helfgott.
\item 2012: supervised \textit{stage de recherche} of E. Naslund (Master 1) at \'Ecole polytechnique; results were presented at the CMS Winter Meeting 2012 and won a student poster prize.
%\item 2012: supervised 3 \textit{projets d'approfondissement} (M1) at \'Ecole polytechnique.
%\item 2011 (summer): supervised stage de recherche of Q. Guyot (masters level) at \'Ecole polytechnique.
%\item 2009: supervised the \textit{independent study project} of junior T. Nguyen at Rutgers University.
\item 2008: supervised research projects of C. Link and Y. Lin at MIT as part of the \textit{Summer Program in Undergraduate Research} (SPUR); both received a departmental award and one resulted in a peer-reviewed publication.
\end{mitemize}

\vspace{3pt}
\textbf{Research degree examination}
\begin{mitemize}
\item 2016: \emph{external Ph.D. examiner} of Przemek Mazur, University of Oxford.
\item 2014: \emph{internal Ph.D. examiner} of T. Bloom, University of Bristol.
\item 2013: \emph{external Ph.D. examiner} of O. Roche-Newton, University of Bristol.
\item 2012: member of the \textit{doctoral thesis committee} of N. de Saxc\'e, Universit\'e Paris-Sud (Orsay).
\end{mitemize}

\vspace{3pt}
\textbf{Teaching}
\begin{mitemize}
\item 2015 (autumn): designed and taught a new 3rd year project unit entitled \textit{Ramsey theory on the integers}; nominated by the department for a University teaching award.
\item 2015 (spring): designed and taught a new 2nd year undergraduate course in \textit{Combinatorics} at the University of Bristol; shortlisted for student-led award for outstanding teaching, Faculty of Science.
\item 2015--present: curriculum oversight and overall responsibility for 3rd/4th year undergraduate course \textit{Topics in discrete mathematics} at the University of Bristol, taught by Heilbronn research fellows.
\item 2014 (autumn): designed and taught a new graduate course entitled \textit{Analysis of points and lines} at the University of Bristol, as part of the Taught Course Centre, transmitted live to Oxford, Imperial, Warwick and Bath.
\item 2014 (spring): \textit{Further topics in analysis}, 1st year undergraduate course, University of Bristol.
\item 2013 (autumn): \textit{Introduction to additive combinatorics}, graduate course, University of Bristol, as part of the Taught Course Centre.
\item 2013--2015: weekly small-group tutorials in pure mathematics, University of Bristol.
\item 2012 and 2013 (spring): designed and taught a new masters course \textit{Combinatoire arithm\'etique et codes} at \'Ecole polytechnique; supervised 9 accompanying projects of Part III-essay type.
\item 2011 and 2013 (summer): petites classes \textit{Analyse r\'eelle et complexe} at \'Ecole polytechnique.
\item 2010 (autumn): petites classes \textit{Distributions, analyse de Fourier et syst\`emes dynamiques} at \'Ecole polytechnique.
\item 2010 (spring): designed and taught a new graduate course entitled \textit{Topics in Probability and Ergodic Theory: Arithmetic structure in the integers and the primes}, and \textit{Calculus II for the Mathematical and Physical Sciences} at Rutgers.
\item 2009 (spring and autumn): \textit{Mathematical Theory of Probability} and \textit{Calculus II for the Mathematical and Physical Sciences} at Rutgers during both semesters.
\item 2008 and 2010 (summer): advanced (multivariable) calculus in the programme \textit{Minority Introduction To Science and Engineering} (MITES) at MIT.
\item 2003--2007: taught around 60 undergraduate students at the University of Cambridge in \textit{Analysis I and II} and \textit{Probability}.
%\item 2005 (autumn): short course for graduate students in discrete mathematics at the  Universitat Polit\`ecnica de Catalunya, Barcelona.
%\item 2001--2002: one of two student representatives on the Teaching Committee of the Faculty of Mathematics in the University of Cambridge.
%\item 2000 (summer): worked for a local charity in Tejalpa, Mexico, teaching English and high-school level mathematics to disadvantaged adults in evening classes and to young children during the day.
\end{mitemize}

\vspace{3pt}
\textbf{Public engagement and outreach}
\begin{mitemize}
%\item February 2016: \textit{Science Caf\'e}, Bristol Branch of the British Science Association.
\item November 2017: \textit{Coulter McDowell Lecture}, Royal Holloway.
\item[$\star$] November 2016: \textit{Opportunities for the future}, co-organiser and speaker, Bristol.
\item June and September 2016: \textit{LMS Popular Lectures}, London and Birmingham.
\item[$\star$] October 2015: \textit{Ada Lovelace Lecture}, public lecture and interview with Bristol 24/7.
\item[$\star$] April 2015: \textit{Interview with journalist David Berry}, submitted to \emph{The Guardian}.
\item December 2014: \textit{Prospects in Mathematics}, University of Oxford.
\item March 2012: \textit{Images des Math\'ematiques}, CNRS.
\item September 2011: \textit{La Nuit des Chercheurs}, \'Ecole polytechnique.
\item November 2009: \textit{Pi-Club}, Rutgers University.
\item November 2009: \textit{Mathematical Careers and Ideas Lecture Series}, Rutgers University.
\item February 2004: \textit{Dilettante Society}, Clare College, Cambridge, with Sir N. Barrington.
\end{mitemize}

\vspace{3pt}
\textbf{Refereeing and reviewing}
\begin{mitemize}
\item 2015--present: Founding editor of the arXiv overlay journal \textit{Discrete Analysis}.
\item 2016--present: Member of the editorial board \textit{INTEGERS}.
\item 2013--present: Editor-in-chief of the \textit{Online Journal of Analytic Combinatorics}.
\item Referee for various journals, including \textit{Proc. Lond. Math. Soc., Trans. Amer. Math. Soc., Proc. Amer. Math. Soc., Adv. Math., Math. Res. Lett., Int. Math. Res. Not., Israel J. Math., Proc. Edinb. Math. Soc., Glasg. Math. J., Collect. Math., Electron. Res. Annouc. Math. Sci., Forum of Mathematics Sigma, Ann. Comb., Finite Fields Appl., J. Combin. Theory Ser. A, J. Comb., Electron. J. Combin., Combinatorica, SIAM J. Comput., J. ACM}.
\item Grant referee for the \textit{US National Security Agency}, the \textit{US-Israel Binational Science Foundation}, the \textit{Israel Science Foundation} and the \textit{French Ministry of Research and Higher Education}.
\item Regular reviewer for \textit{Mathscinet} (36) and \textit{Zentralblatt MATH} (22).
\end{mitemize}

%\vspace{10pt}
%\textbf{Research grants and other funding}
%\begin{mitemize}
%%\item 2012: submitted \emph{ERC Starting Grant} TAHOFA (EUR 1,153K).
%%\item 2012: submitted \emph{FACCTS France-Chicago Grant} APSAA, with M. Tulsiani (USD 10K).
%\item 2012--2015: awarded grant from the \textit{Agence Nationale de la Recherche, programme Blanc} 
%\\CAESAR (EUR 203K), with E. Breuillard, A. Granville, H. Helfgott, A. Plagne et al.
%%\item AMS ICM travel grant, 2010 [declined].
%\item 2010: \textit{AWM mathematics travel grant}.
%\item 2003--2007: \textit{Gates Cambridge scholarship}.
%\item 2003--2007: \textit{EPSRC research studentship}.
%\end{mitemize}

\vspace{3pt}
\textbf{Other professional activities}
\begin{mitemize}
%\item Co-director of the \textit{masters programme} in mathematics at \'Ecole polytechnique, 2013--.
\item 2015--present: Associate chair, \textit{Heilbronn Institute for Mathematical Research}, with budget responsibility for funds within the School of Mathematics, leading mentoring and career-development activities for $\sim$ 25 Heilbronn research fellows.
\item 2016: co-author of \emph{EPSRC landscape document in Combinatorics}, with D. Kuhn and D. Kral.
\item 2012--2013: Coordinator \textit{stages de recherche} for masters MAT-INF at \'Ecole polytechnique.
\item 2012--2015: Elected member of the \emph{department committee} at \'Ecole polytechnique.
\item2012--2015, 2015--2018: Elected member of the \textit{Board of Trustees} of the \emph{Association of Members of the Institute for Advanced Study}, Princeton.
\item Member of the \textit{London Mathematical Society}.
\item[$\star$] Member of \textit{European Women in Mathematics}.
\end{mitemize}

%\vspace{3pt}
%\textbf{References}
%
%\begin{mitemize}
%\item \textit{Terence Tao}\\
%Email: tao@math.ucla.edu, Phone: +1 310 206 4844\\
%Department of Mathematics,
%University of California, Los Angeles, CA 90095.
%\item \textit{Endre Szemer\'edi}\\
%Email: szemered@cs.rutgers.edu, Phone: +1 848 445 7295\\
%Department of Computer Science,
%Rutgers University,
%Piscataway, NJ 08854.
%\item \textit{Timothy Gowers}\\
%Email: w.t.gowers@dpmms.cam.ac.uk, Phone: +44 1223 337973\\
%Department of Pure Mathematics and 
%Mathematical Statistics,
%Cambridge CB3 0WB.
%\end{mitemize}

\end{document}




